\documentclass[final,hyperref={pdfpagelabels=false}]{beamer}
%\usepackage{grffile}
\mode<presentation>{
\usetheme{cbsposter}
}
\usepackage[english]{babel}
\usepackage[latin1]{inputenc}
\usepackage{amsmath,amsthm, amssymb, latexsym}
%\usepackage{times}\usefonttheme{professionalfonts}  % obsolete
%\usefonttheme[onlymath]{serif}
\boldmath
\usepackage[orientation=portrait,size=a0,scale=1.4,debug]{beamerposter}
\usepackage[numbers]{natbib}
% change list indention level
% \setdefaultleftmargin{3em}{}{}{}{}{}


%\usepackage{snapshot} % will write a .dep file with all dependencies, allows for easy bundling
%\usepackage{array,booktabs,tabularx}

%\listfiles

%%%%%%%%%%%%%%%%%%%%%%%%%%%%%%%%%%%%%%%%%%%%%%%%%%%%%%%%%%%%%%%%%%%%%%%%%%%%%%%%%%%%%%
\graphicspath{{figures/}}
 
\title{\huge Lsh, nearest neighbor search in high dimensions}
\author{Edwin de Jonge}
\institute[Statistics Netherlands]{Methodology Department, Statistics Netherlands, Den Haag, Netherlands}
\date[July. 2nd, 2014]{July 2nd. 2014}

%%%%%%%%%%%%%%%%%%%%%%%%%%%%%%%%%%%%%%%%%%%%%%%%%%%%%%%%%%%%%%%%%%%%%%%%%%%%%%%%%%%%%%
\newlength{\columnheight}
\setlength{\columnheight}{105cm}

%%%%%%%%%%%%%%%%%%%%%%%%%%%%%%%%%%%%%%%%%%%%%%%%%%%%%%%%%%%%%%%%%%%%%%%%%%%%%%%%%%%%%%
\begin{document}
\begin{frame}[fragile]
  \begin{columns}
    % ---------------------------------------------------------%
    % Set up a column 
    \begin{column}{.49\textwidth}
      \begin{beamercolorbox}[center,wd=\textwidth]{postercolumn}
        \begin{minipage}[T]{.95\textwidth}  % tweaks the width, makes a new \textwidth
          \parbox[t][\columnheight]{\textwidth}{ % must be some better way to set the the height, width and textwidth simultaneously
            % Since all columns are the same length, it is all nice and tidy.  You have to get the height empirically
            % ---------------------------------------------------------%
            % fill each column with content            
            \begin{block}{Introduction}
            Data sets with many variables ($p$) and rows ($n$) nowadays 
            very common. 
            
                \begin{itemize}
                \item $p$ can run from $10-1000$ variables. 
                \item $n$ runs from $10^3$ to $10^6$.
                \item Curse of dimensionality
              \end{itemize}
            
            ~\\
            Traditional nearest neighbor techniques:            
              \begin{itemize}
             \item Both a large $n$ and $p$ are challenging.
             \item Calculating distance pairs is $O(n^2)$ 
             \item Finding the nearest neighbor is $O(n)$ in time.
             \item kd-tree~\citep{bentley1975} was developed to cope with large $n$, but performance
             down for $p > 3$~\citep{datar2004}. 
              \end{itemize}
            
            \end{block}
            \vfill
            \begin{block}{LSH}
{\bf Locality sensitive hashing (LSH)~\citep{datar2004}}
\begin{itemize} 
\item a technique for generating hash numbers
from high dimensional data
\item nearby points identical hashes. Map each observation/data point onto an integer.
\item enables efficient nearest neighbor search for (very) high dimensional data sets.
\item Is an approximated, randomized algorithm with high probability guarantee correct answer will be found. 
\end{itemize}~\\
  Applications:
  \begin{itemize}
    \item Image search: bitmaps ($1000$ x $1000$) pixels, form a $10^6$ vector.
    \item Duplicate web pages~\citep{slaney2008}
    LSH has been applied successfully to several problems including text similarity search for finding 
  \end{itemize}
            \end{block}
            \vfill
            \begin{block}{Random projection}
            At the core of LSH is scalar projection:
            $$
            h (\vec{v})=  \frac{\vec{x} \cdot \vec{v} + b}{w}
            $$
            %$$h(\vec{v}) = \vec{x} \ldot \vec{v} + b$$
              \begin{itemize}
              	\item Family of $L$ hash function
              	\item
              \end{itemize}
            \end{block}
            \vfill
            \begin{block}{Hash functions}
              \begin{itemize}
              	\item Family of $L$ hash function
              	\item
              \end{itemize}
            \end{block}
          }
        \end{minipage}
      \end{beamercolorbox}
    \end{column}
    % ---------------------------------------------------------%
    % end the column

    % ---------------------------------------------------------%
    % Set up a column 
    \begin{column}{.49\textwidth}
      \begin{beamercolorbox}[center,wd=\textwidth]{postercolumn}
        \begin{minipage}[T]{.95\textwidth} % tweaks the width, makes a new \textwidth
          \parbox[t][\columnheight]{\textwidth}{ % must be some better way to set the the height, width and textwidth simultaneously
            % Since all columns are the same length, it is all nice and tidy.  You have to get the height empirically
            % ---------------------------------------------------------%
            % fill each column with content
            \begin{block}{Distance functions}
            {\bf\Large Different distance functions:}
            \begin{itemize}
             \item Euclidean: \\
             Original algorithm: E$^2$LSH.\\
             Useful for numeric data
             \item $L^p$ distance, including Manhattan etc.
             \item Hamming Distance \\
             Useful for categorical, binary data
             \item Jaccard Distance \\
             Useful for textual data.
            \end{itemize}
            \end{block}
            \vfill
            \begin{block}{Implementation}
              Implemented in {\tt R}:
              \begin{itemize}
                \item package {\tt lsh} on {\tt http://github.com/edwindj/lsh}~\\
                \item {\tt LSHIndex(data, L=10, k=10, w=32, distance=c("euclidean"))} ~\\
                  \begin{itemize}
                    \item {\tt data}: data to be indexed
                    \item {\tt L}: number of hash functions $h_i$ for each $g_k$
                    \item {\tt k}: number of $g_k$'s
                    \item {\tt w}: size of quantizer w (bins)
                    \item Reference object that contains all hashes for data
                    \item {\tt \$query(data, distance=c("euclidean")}
                  \end{itemize}~\\
                \item {\tt\large knn\_lsh(data, k=10, lsh\_index=NULL, distance=NULL)}.
                 \begin{itemize}
                   \item {\tt data}: data for which {\tt k} nearest neighbors are found
                   \item {\tt k}: number of nearest neighbors
                   \item {\tt lsh\_index}: index to be used for searching neighbors. If {\tt NULL} lsh\_index will be created from {\tt data}.
                   \item {\tt distance}: distance function, by default euclidean distance. 
                 \end{itemize}
              \end{itemize}
            \end{block}\vfill
            \begin{block}{Work in progress}
            {\emph Still in heavy development. Feedback more than welcome!}
            \end{block}
            \vfill
            \begin{block}{References}
\nocite{*}
\bibliographystyle{chicago}
\bibliography{lsh}

            \end{block}
          }
          % ---------------------------------------------------------%
          % end the column
        \end{minipage}
      \end{beamercolorbox}
    \end{column}
    % ---------------------------------------------------------%
    % end the column
  \end{columns}
  \vskip1ex
  \tiny\hfill{Created with \LaTeX \texttt{beamerposter}  \url{http://www-i6.informatik.rwth-aachen.de/~dreuw/latexbeamerposter.php} \hskip1em}
\end{frame}
\end{document}


%%%%%%%%%%%%%%%%%%%%%%%%%%%%%%%%%%%%%%%%%%%%%%%%%%%%%%%%%%%%%%%%%%%%%%%%%%%%%%%%%%%%%%%%%%%%%%%%%%%%
%%% Local Variables: 
%%% mode: latex
%%% TeX-PDF-mode: t
%%% End:
